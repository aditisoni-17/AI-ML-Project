\documentclass[conference]{IEEEtran}
\usepackage[utf8]{inputenc}
\usepackage{amsmath, amssymb, amsthm}
\usepackage{graphicx}
\usepackage{hyperref}
\usepackage{booktabs}
\usepackage{cite}

\title{Intelligent Credit Risk Scoring \& Agentic Lending Decision Support System}

\author{\IEEEauthorblockN{Krit Garg, Divya Pahuja, Deepak Pathik, Aditi Soni}
\IEEEauthorblockA{Horizon}}

\begin{document}

\maketitle

\begin{abstract}
This document presents an end-to-end AI-powered fintech platform designed to predict borrower credit risk. Traditional underwriting processes are often rigid and lack transparency. Our system introduces a two-phase architecture: a quantitative Machine Learning pipeline for rapid prediction of default probabilities, and a qualitative Agentic AI Assistant for formulating professional, explainable lending recommendations. Using a Random Forest classifier combined with robust preprocessing, the system accurately classifies high-risk and low-risk candidates while prioritizing critical metrics such as Recall and ROC-AUC.
\end{abstract}

\begin{IEEEkeywords}
Credit risk, machine learning, random forest, explainable AI, agentic systems, fintech.
\end{IEEEkeywords}

\section{Introduction}
Traditional credit underwriting relies heavily on static scorecards and rigid, rule-based systems. These legacy systems often fail to adapt to complex financial behaviors and lack transparency into why an applicant was rejected. 

The purpose of this platform is to solve the manual bottleneck in modern fintech by introducing a system that can continuously learn and adapt. The architecture enables scaling loan origination, reducing human underwriting time, and maintaining explainable compliance standards through a hybrid ML and Agentic approach.

\section{Methodology}

\subsection{Dataset Description}
The project utilizes a Credit Risk Benchmark Dataset representing historical profiles of borrowers. Key features include age, monthly income, number of dependents, utilization of credit limits (Revolving Utilization), debt ratios, and historical repayment behavior. The target variable constitutes a binary flag indicating whether a borrower hit delinquency within a 2-year window.

\subsection{Preprocessing \& Engineering}
To maintain dataset integrity, missing values in numerical features were handled via automated median imputation, staying robust to extreme outliers typical in income data. Variables are uniformly scaled to prepare for model ingestion.

\subsection{System Architecture}
The basic architecture of the integrated scoring system is shown in Figure \ref{fig:arch}. It follows a linear pipeline from ingestion to risk classification.

\begin{figure}[h]
    \centering
    \includegraphics[width=0.48\textwidth]{architecture.png}
    \caption{AI Credit Risk Scoring System Architecture}
    \label{fig:arch}
\end{figure}

\subsection{Modeling Algorithm}
The primary model is a \texttt{RandomForestClassifier}. The model calculates the predicted class $\hat{y}$ for an applicant $x$ by aggregating the predictions from $B$ individual decision trees $T_b$:
\begin{equation}
    \hat{y} = \text{mode} \{ T_1(x), T_2(x), \dots, T_B(x) \}
\end{equation}
Furthermore, the default probability $P(Y=1 | X)$ is estimated as:
\begin{equation}
    P(Y=1 | X) = \frac{1}{B} \sum_{b=1}^{B} P_b(Y=1 | X)
\end{equation}

To combat class imbalance, the system uses the \texttt{class\_weight="balanced"} hyperparameter with 100 base estimators.

\section{Results}

\subsection{Performance Metrics}
The model was validated using an 80/20 train-test split. The results, as tested on the hold-out set, demonstrate robust separation ability:

\begin{table}[h]
\centering
\caption{Model Performance Metrics}
\label{table_metrics}
\begin{tabular}{lc}
\toprule
Metric & Random Forest Score \\
\midrule
Accuracy & 0.7751 \\
Precision & 0.7750 \\
Recall & 0.7750 \\
ROC-AUC & 0.8531 \\
\bottomrule
\end{tabular}
\end{table}

\subsection{Feature Importance (XAI)}
The Random Forest model inherently estimates the importance of each feature in reducing node impurity. Figure \ref{fig:importance} provides a visualization of the top factors contributing to the risk scoring outcome.

\begin{figure}[h]
    \centering
    \includegraphics[width=0.48\textwidth]{importance.png}
    \caption{Model Feature Importance - Primary Risk Indicators}
    \label{fig:importance}
\end{figure}

As shown, \textbf{Revolving Utilization} and \textbf{Debt Ratio} emerge as the leading indicators of financial stress and credit risk.

\section{Conclusion}
This system demonstrates the practical viability of hybridizing predictive machine learning with generative AI in fintech. By achieving strong ROC-AUC performance and providing structured recommendations, the platform sets a modern standard for efficient, interpretable credit underwriting.

\appendix
\section{Repository Layout}
The complete source code and trained models follow standard logical groupings:
\begin{itemize}
    \item \texttt{app/}: Streamlit UI scripts (\texttt{streamlit\_app.py}).
    \item \texttt{src/}: Cleaning logic and main training entry.
    \item \texttt{models/}: Serialized model files.
    \item \texttt{README.md}: Project overview and setup.
\end{itemize}

\end{document}
